\documentclass{article}
\usepackage{verbatim}
\usepackage{listings}
\usepackage{footnote}
\makesavenoteenv{tabular}

\newcommand{\code}[1]{\texttt{#1}}
\newcommand{\codeblock}[1]{\begin{quote}\code{#1}\end{quote}}

\newcommand{\name}{\code{41++}}

\begin{document}
\title{Manual for the \name Language}
\author{Kavi Gupta}
\maketitle

\section{The Concept of Programming in \name}
\name{} is a language that allows people to talk to computers in a language any human should be able to understand. The language uses simple words and phrases to describe what are, in fact, simple concepts that many languages confuse with symbols. \name{} is \emph{not} the most useful language on the planet for writing airplane control systems, payroll distribution systems, or other programming languages (although we \emph{will} implement a compiler in the future to demonstrate some of \name{}'s features). Instead, \name{} is intended to be a programming language so that beginners can learn the basics of algorithms, functions, and structures in a text-based yet simple format.\footnote{If you don't know what any of that just meant, don't worry---just read the rest of this manual to find out!}

\end{document}
