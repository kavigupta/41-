\documentclass{article}
\usepackage{verbatim}
\usepackage{listings}
\usepackage{footnote}
\usepackage[margin=1in]{geometry}
\makesavenoteenv{tabular}
\usepackage{color}
\usepackage{soul}
\usepackage{calc}
\usepackage[framemethod=default]{mdframed}

\newcommand{\code}[1]{\texttt{#1}}
\newcommand{\codeblock}[1]{\begin{quote}\code{#1}\end{quote}}

\newmdenv[linecolor=white,backgroundcolor=black]{blackbox}
\newenvironment{console}{\begin{blackbox}\code\bgroup\color{white}}{\egroup\end{blackbox}}

\newcommand{

\newcommand{\name}{\code{41++}}
\newcommand{\ide}{\name{} Editor}
\newcommand{\con}{\name{} Console}

\newcommand{\enter}{``ENTER''}

\newcommand{\consExpression}[2]{
\begin{console}
\begin{tabular}{ll}
>> &#1\\
\textcolor{yellow}{\hspace*{1ex}=} &\textcolor{yellow}{#2}
\end{tabular}
\end{console}}

\newcommand{\consInput}[1]{\textcolor{white}{\colorbox{black}{\code{#1}}}}
\newcommand{\consExpr}[1]{\textcolor{yellow}{\colorbox{black}{\code{#1}}}}

\begin{document}
\title{Manual for the \name Language}
\author{Kavi Gupta}
\maketitle

\section{The Concept of Programming in \name}
\name{} is a language that allows people to talk to computers in a language any human should be able to understand. The language uses simple words and phrases to describe what are, in fact, simple concepts that many languages confuse with symbols. \name{} is \emph{not} the most useful language on the planet for writing airplane control systems, payroll distribution systems, or other programming languages (although we \emph{will} implement a compiler in the future to demonstrate some of \name{}'s features). Instead, \name{} is intended to be a programming language so that beginners can learn the basics of algorithms, functions, and structures in a text-based yet simple format.\footnote{If you don't know what any of that just meant, don't worry---just read the rest of this manual to find out!}

\section{The Console}
This section will make a lot more sense if you have \ide{} open, so I suggest you open it now.

The console is the little box at the bottom, marked, as you might suspect, ``\con{}.'' You can write stuff into it and press \enter{}. The usefullness of the console will be discussed later in this section.

\section{Literals}

Try typing \consInput{2} into your console. You should get something that looks like this:

\consExpression{2}{2}

Notice how the second line is in yellow; this, along with the equals sign before it, signifies that it is an output.

Now, you may be thinking ``OK, that's great, it's a duplication program. Very very useful. [end sarcasm]'', but in fact explaining this simple characteristic will give you some insight into how to use \name{}. When you enter \consInput{2} into the document, then the \name{} interpreter calculates the literal value of the input, which in this case is helpfully \consExpr{2}.

Anything that when typed into the console outputs itself is known as a ``literal''. Numbers are an example of such a literal. Valid numbers consist of a sequence of digits \code{0-9}, with an optional period that must be followed by more digits\footnote{Some languages allow for numbers like \code{4.}, but periods have a special meaning in \name{} so you must write \code{4.0} or just \code{4} (no period at the end).}

The second type of literals are strings. Strings are slightly more complicated than numbers, but the simple rule is that they must start and end with a single quote (\code{'}). Here are a few examples:

\consExpression{'abc'}{'abc'}
\consExpression{'1 2 3'}{'1 2 3'}
\consExpression{'In \textbackslash{}'quotes\textbackslash{}''}{'In 'quotes''}

As you can see, the text isn't exactly repeated. For example, \code{\textbackslash{}'} becomes a quote. This is the only way of inserting a quote into a literal string, known as an ``escape''. Other valid escapes are as follows:

\begin{tabular}{|l|l|
\hline
Sequence & Value\\
\code{\backslash{}n} & A new line, the equivalent of the \enter{} key\\
\end{tabular}

\end{document}
