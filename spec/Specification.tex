\documentclass{article}
\usepackage{verbatim}
\usepackage{listings}
\usepackage{footnote}
\makesavenoteenv{tabular}

\newcommand{\code}[1]{\texttt{#1}}
\newcommand{\codeblock}[1]{\begin{quote}\code{#1}\end{quote}}
 
\begin{document}
\title{Specification for the 41++ Language}
\author{Kavi Gupta}
\maketitle
\section{Types}
\subsection{Primitives}
Primitive types are types that do not have fields. Instead, they merely have values. These can take the form:
\begin{itemize}
\item \code{void}: a type representing a the return type of a function that does not return a value. It has no enumerated values and variables of type \code{void} cannot be created.
\item \code{number}: a numeric type. This includes both integers and floating point numbers. Calculations internally are handled using high precision, large numbers.
\item \code{character}: a character type. This is basically an integer that will be shown as a character.
\item \code{bool}: a boolean type. \code{true} or \code{false}.
\item \code{type}: a type type. This type is the type of the \emph{values} (not types!) \code{number}, \code{type}, \code{array of string}.
\end{itemize}
\subsection{Defined types}
Defined types are the types of structures with fields. These fields can be themselves defined types or primitives.
\subsection{Compound types}
Compound types are types that take other types as arguments, similar to Haskell's type constructors. A common example is \code{array} which can only be declared as \code{array of T}, where \code{T} is the type stored in the array. Compound types are always defined types, because types are fields.
\section{Naming}
\subsection{Forbidden Characters}
Characters whose purpose is so specialized that they cannot be used anywhere else (except in string and character literals) are called forbidden. Here is a complete list.
\begin{itemize}
\item \code{(}, \code{)}, \code{[}, \code{]}
\item Algebraic Expressions
\item \code{\textbackslash}\footnote{Must be escaped in character and string literals}
\end{itemize}
\subsection{Punctuation}
Punctuation in 41++ consists of \code{.}, \code{,}, \code{:}, and \code{;}. No identifier can contain punctuation at the end, although it is permissible internally.
\subsection{Literals}
As seen above, literals start with a number or \code{'}. Collectively, these are called literal flags.
\subsection{Variables}
Although variables without prefixes would normally be beneficial in a natural-like language like 41++, they could possibly cause ambiguity with Arbitrary Syntax Functions. Therefore, all variables must start with the underscore character \code{\_} and cannot contain whitespace.
\subsection{Functions and Structures}
Functions and structures cannot have any non-variable tokens in their name start with a parenthesis, underscore, or literal flag.
\section{Functions}
\subsection{Declaration}
A function is declared as follows:

\codeblock{Define a function called <function expression> that takes a[n] <type1> called <field1>, a[n] <type2> called <field2>, and a[n] <type3> called <field3>[ and outputs a <return type>].}

Where, in the function expression, variable names are enumerated as they appear in the type declarations later. Note that the function expression doesn't actually have a defined syntax. For example, this is a valid declaration:

\codeblock{Define a function is \_{}n prime that takes a number called \_{}n and outputs a bool.}

Note that if \code{ and outputs a <return type>} is omitted, then the return type defaults to \code{void}.

\subsection{Body}
The body consists of statements, in which any variables that are made cannot be used outside the function.
\subsection{Conclusion}
A function return is declared as follows.

\codeblock{Exit the function[ and output <output>].}

Note that the output must be omitted if the return type is \code{void} because \code{void} has no values.

\section{Structures}
\subsection{Declaration}
A structure is declared as follows:

\codeblock{Define a structure called <structure expression>; which contains a[n] <type1> called <field1>, a[n] <type2> called <field2>, and a[n] <type3> called <field3>.}

The structure expression may contain variables, whose \code{type} is automatically \code{type}. A concrete example is probably best in this situation.

\codeblock{Define a structure called \_{}k mapped to \_{}v; which contains an \_{k} called key and a \_{v} called value.}

This defines an entry structure that can be later defined as follows.

\codeblock{Define a (number mapped to number) entry with a key of 2 and a value of 2.}

Note the use of defined types to replace \code{\_{a}} and \code{\_{b}}.

\subsection{Field Access}

Field access is simple, taking the following syntax.

\codeblock{the <field> of <structure>}

\subsection{Arrays}
Arrays are, in a sense, a primitive structure. To declare an array, the following syntax is used:

\codeblock{Define a[n] array of <type> called <name> with a size of <size>.}

Array access and definition are treated as if defined as such, with one definition for every type \code{<T>} in existence.

\codeblock{Define a function called the \_{n} th element of \_{array} that takes a number called n and an (array of <T>) called array and outputs a <T>.}

\codeblock{Define a function called Set the \_{n} th element of \_{array} to \_{value} that takes a number called n, an (array of <T>) called array, and a <T> called \_{}value.}

Note: Technically, three different functions are defined, with two replacing \code{th} with \code{rd} and \code{st}, to prevent expressions such as \code{the 2 th element of \_{}array1}.

\section{Expressions}
\subsection{Literals}
Literals are expressions that are hard-coded into the code. They take one of four forms.
\subsection{Numeric Literals}
These must start with a digit, a plus sign, a minus sign, or a period.
\subsection{Boolean Literals}
Either \code{true} or \code{false}.
\subsection{Character and String Literals}
These must start and end with a single quote \code{'}. What is in between is interpreted as a string. To use an actual single quote mark, use \code{\textbackslash'}. Standard escapes can also be used. Determining the type of a string falls into three cases.
\begin{itemize}
\item \code{''}: This is automatically a string literal representing an empty string.
\item A single character: depending on the context, this is interpreted as a string or character.
\item Multiple characters: always a string.
\end{itemize}
\subsubsection{Examples}
\begin{itemize}
\item \code{+2}, \code{-56543234565}, \code{41}, \code{-.02345654321}, \code{12.0}: Numeric literals.
\item \code{'"'}, \code{'1'}, \code{'\textbackslash{}r'}, \code{'\textbackslash{}n'}, \code{'\textbackslash{}t'}, \code{'\textbackslash{}0123'}: Character literals
\item \code{''}, \code{'\textbackslash'\textbackslash''}, \code{'41++'}: String literals.
\end{itemize}
\subsection{Algebraic Expressions}
\begin{tabular}{|l|p{10cm}|}
\hline
Algebraic Expression & Type\\
\hline
\code{=} & \code{a = a -> bool}\\
\hline
\code{/=} & \code{a = a -> bool}\\
\hline
\code{>} & \code{number > number -> bool} and \code{char > char -> bool}\\
\hline
\code{<} & \code{number < number -> bool} and \code{char < char -> bool}\\
\hline
\code{>=}& \code{number >= number -> bool} and \code{char >= char -> bool}\\
\hline
\code{<=} & \code{number <= number -> bool} and \code{char <= char -> bool}\\
\hline
\code{+}& \code{number + number -> number} and \code{char + char -> char} and \code{char + number -> number}\\
\hline
\code{-} & \code{number - number -> number} and \code{char - char -> char} and \code{char - number -> number}\\
\hline
\code{*} & \code{number * number = number}\\
\hline
\code{/} & \code{number / number = number}\footnote{this is standard division. \code{11/2 = 5.5}}\\
\hline
\code{//} & \code{number // number = number}\footnote{this is floor division. \code{11/2 = 5, 1.5 / 1 = 1, -1.5 / 1.2 = -1}}\\
\hline
\code{\%} & \code{number \% number = number}\footnote{remainder}\\
\hline
\end{tabular}
\subsection{Function Expressions}
Any function call of a non-void returning function can be used as an expression. This includes structure and array access functions.
\subsection{The Role of Parentheses}
While 41++ is designed to be a English-like language, it is often very difficult to tease out syntactical ambiguity without parentheses. Therefore, in 41++, parentheses must surround any value that is not a single word or string literal.
\section{Statements}
There are a limited number of valid statement forms. All start with a capital letter and end with a period.
\subsection{Definition}
\subsubsection{Declaration}
A minimal declaration simply provides a variable with a name and associates it with a type.

\codeblock{Define a[n] <type> called <name>.}

\subsubsection{Field Initialization}
A variable can also have its fields initialized. It can also be directly set to a value by using the special field \code{value}.

\codeblock{Define a[n] <type> called <name> with a[n] <field1> of <value1>, a[n] <field2> of <value2>, and a[n] <field3> of <value3>.}

\noindent Commas and \code{and} are all technically unnecessary, but included to insure readability. Similarly, \code{a} and \code{an} are equivalent but both are included to avoid statements like \code{Define a integer called x.}
\subsubsection{Examples}

\codeblock{Define an integer called \_{}x. Define a string called \_{}name with a value of '41++'. Define a matrix called \_{}M with a width of 3 and a height of 2. Define a matrix called \_{}M2 with a value of \_{}M.}

\subsection{Assignment}
Assignment comes in two forms, value assignment and field assignment.

\codeblock{Set the <field> of <name> to <value>.}

\subsection{Function Calls}
Any function call can be a statement in one of these two ways.

\begin{itemize}
\item \codeblock {Run <non-void function expression>.}
\item \codeblock {<Void expression>.}
\end{itemize}

\section{Control Flow}
\subsection{If}
The syntax of If is as follows.
\codeblock{If <expression>: <Statement executed in case of expression>.}
An Otherwise block can also be appended.
\codeblock{If <expression>: <Statement executed in case of expression>; otherwise: <Statement executed otherwise>.}
\subsection{While}
The syntax of While is as follows.
\codeblock{While <expression>: <Statement>.}
\subsection{Statement Concatenation}
Notice that control flow statements above only take a single statement as an argument. However, having to define a procedure for every small set of instructions would be difficult. Instead, we can use the following syntax to convert multiple statements into a single one:
\codeblock{<First statement>; <Second statement>; <Third statement>.}
\section{Comments}
Comments are sections that are not to be interpreted as code. The syntax is as follows.
\codeblock{[<Ignore whatever goes here>]}
\end{document}